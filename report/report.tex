\documentclass[12pt]{article}
\usepackage[utf8]{inputenc}
\usepackage{amsmath, amssymb}
\usepackage{algpseudocode}
\usepackage{hyperref}
\usepackage{amsthm}
\usepackage{xcolor}

\setlength\parindent{0pt}
\newcommand\note[1]{\textcolor{red}{\textbf{#1}}}


\title{Parallel Odd-Even Sort}
\author{Lorenzo Beretta (\texttt{lorenzo2beretta@gmail.com})}
\date{6th June 2020}

\begin{document}
\maketitle

\section{Introduction}
Sorting is a basic building block (bla bla...). Sequential algorithm
complexity, why is it a good idea to parallelize this algo (strong
data independence).
Related work: i.e. why didn't I go further? There are plenty of
options to improve over my results but I decided to stick to the
original implementation without relying, for example, on O(n log(n))
sequential sorter as building blocks of my algo.

\section{Results Overview}


\bibliographystyle{acm}  
\bibliography{biblio}
\end{document}

%%% Local Variables:
%%% mode: latex
%%% TeX-master: t
%%% End:
